%% LyX 1.2 created this file.  For more info, see http://www.lyx.org/.
%% Do not edit unless you really know what you are doing.
\documentclass[english]{article}
\usepackage[T1]{fontenc}
\usepackage[latin1]{inputenc}
\usepackage{geometry}
\geometry{verbose,letterpaper,tmargin=1in,bmargin=1in,lmargin=1in,rmargin=1in}

\makeatletter

%%%%%%%%%%%%%%%%%%%%%%%%%%%%%% LyX specific LaTeX commands.
\providecommand{\LyX}{L\kern-.1667em\lower.25em\hbox{Y}\kern-.125emX\@}

\usepackage{babel}
\makeatother
\begin{document}

\title{Homework and Midterm Review\\
CSC4200/5200 \date{}}

\maketitle
\textbf{Due: March 18, 2003}

\begin{enumerate}
\item \_\_\_\_\_\_\_\_\_\_\_\_\_\_\_\_\_\_ is the protocol suite for the
current Internet.
\item Why is layering used to describe and engineer network software?

\begin{enumerate}
\item Layering makes network protocols faster
\item Layering ensures that protocols stacks implemented by different agencies
can communicate with each other
\item Layering is necessary to enable to software to interface with the
hardware (i.e. network cards)
\item Layering naturally categorizes the functionality of hierarchical protocol
suites into modules, or \emph{layers}, that help define standards
and ease engineering and maintenance.
\end{enumerate}
\item Which of the following are layers in the TCP/IP suite?

\begin{enumerate}
\item Physical
\item Network
\item Logical
\item Transport
\item Link
\item Intermediate
\item Protocol
\item Virtual circuit
\end{enumerate}
\item As a data packet moves from the lower to the upper layers, headers
are \_\_\_\_\_\_\_\_\_\_\_\_

\begin{enumerate}
\item added
\item subtracted
\item rearranged
\item modified
\end{enumerate}
\item What is the main function of the transport layer?

\begin{enumerate}
\item router-to-router delivery
\item process-to-process delivery
\item synchronization
\item updating and maintenance of routing tables\newpage
\end{enumerate}
\item Which of the following is an application layer service?

\begin{enumerate}
\item file transfer
\item email
\item nameserver
\item all of the above
\item (a) and (b)
\end{enumerate}
\item Which layer is responsible for each of the following?

\begin{enumerate}
\item route determination \_\_\_\_\_\_\_\_\_\_\_\_\_\_\_\_\_\_\_\_\_\_\_\_\_\_\_\_\_\_\_\_\_\_\_
\item flow control and reliability \_\_\_\_\_\_\_\_\_\_\_\_\_\_\_\_\_\_\_\_\_\_\_\_\_\_\_\_\_
\item interface to the outside world \_\_\_\_\_\_\_\_\_\_\_\_\_\_\_\_\_\_\_\_\_\_\_\_\_\_\_
\item point-to-point data transfer\_\_\_\_\_\_\_\_\_\_\_\_\_\_\_\_\_\_\_\_\_\_\_\_\_\_\_\_\_\_
\item transfering bits down the wire\_\_\_\_\_\_\_\_\_\_\_\_\_\_\_\_\_\_\_\_\_\_\_\_\_\_\_
\end{enumerate}
\item Suppose transmission channels become virtually error free. Is the
data link layer still needed? Why?
\item Why is the transport layer not present inside the network?
\item Suppose a data link layer provides reliable delivery from one hop
to the next. Is this sufficient to provide reliable delivery between
hosts? why or why not?
\item Why do HTTP, FTP, and SMTP run over TCP rather than UDP?
\item Describe the following (tell what they do):

\begin{enumerate}
\item Hub
\item Bridge
\item Router
\end{enumerate}
\item Answer the following questions from the book in chapter 2 under Review
Questions: (9), (11), (12), (13), (16), (17) 
\item Solve problems (6) and (7) from the \textbf{Problems} section in chapter
2.\end{enumerate}

\end{document}
