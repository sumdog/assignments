\documentclass[a4paper,11pt]{article}
\author{Sumit Khanna}
\title{CSC 4200 Homework 2}

\begin{document}
\maketitle
\section{Chapter 3 - Review Questions}
\paragraph{}1) They will simply be the opposite ports. source=y and dest=x. The source and destination are always relative to the computer who's sending.

\paragraph{}4a) False. Host B will attempt to piggyback an ACK onto a packet in its queue that's intended for host A. If it searches its queue and can't find another packet that's destined for that host, it will simply send a new packet which only contains the ACK

\paragraph{}4b) False. The receiver has a buffer used to pull packets together in the correct order. Say packets 1, 2, 3, 7 and 8 come it. It will ship 1, 2 and 3 up to the application as soon as they come in because they're in order. 7 will be buffered until packets 4, 5 and 6 come in at which time packets 4 through 8 will be sent to the application layer. The buffer is of a fixed size. The amount of free space in the buffer is known as the RcvWindow and will, of course, change and the buffer is filled and emptied. It's important not to confused the RcvWindow, which is just an indicator on how full a host's buffer is, with the actual Buffer which is a static allocation of memory (non-changing).

\paragraph{}4c) True. According to TCP, you should not send any packets until you're sure there's room in the receiver's buffer. However, you should still send empty packets as acknowledgments that you're still there sending data.

\paragraph{}4d) False. In TCP, the segment number indicates the number of byes you're at in the stream. If I send 2k of data (2k = 2000) starting with sequence number 40, then the next sequence number will be 40 + 2000. 

\paragraph{}4e) True (p 231)

\paragraph{}4f) False TimeoutInterval = EstimatedRTT + 4 * DevRTT (p 237)

\paragraph{}4g)

\section{Chapter 3 - Problems}

\paragraph{}1) 
\\a) Source: 5002 Dest: 22
\\b)Source: 6012 Dest: 22
\\c) Source: 22 Dest: 5002
\\d) Source: 22 Dest: 6012

\section{Chapter 4 - Questions}
\paragraph{}RIP - A distance vector protocol with a maximum cost path of 15 (with each link being 1), limiting the use of RIP to autonomous systems that are fewer than 15 hops in diameter. Updates are done between neighbors approximately every 30 seconds. (p 346)

\paragraph{}OSPF: Open Shortest Path First - Link State Protocol that uses flooding of link state information and a Dijkstra least-cost path algorithm. (p 350)

\paragraph{}BGP: Border Gateway Protocol - de facto standard inter-domain routing protocol in today's Internet. It's a path vector protocol that exchanges detailed path information between neighboring routers.


\paragraph{}IPv4 - IPv6 - The current version of IP used in todays networks is IPv4. IPv4 consists of four numbers in base 256 separated by periods (e.g. 149.149.30.18). IPv4 is limited by the fact that the allocation space for IP numbers is only 32-bits long. Because of this, we are quickly running out of IP addresses for the ever growing Internet. Also, certain ranges are reserved in IPv4 (192.168.x.x and 10.x.x.x) for private network IP addresses. The successor to IPv4 is IPv6. IPv6 uses a 128-bit space for allocating numbers allowing for a much larger space from which to allocate numbers. IPv6 also supports placing certain priority on packets and routing based on Quality of Service. The disadvantages of IPv6 is that the current layer of TCP can't utilize all the IPv6 functionality. Also, there is no slated "conversion date" when everyone will switch over (Japan has officially announced that IPv6 must be fully implement by 2006 on their networks). IPv6 address also look nothing like IPv4 addresses (numbers are in base 16/Hex).

\end{document}
